\documentclass{book}

\usepackage[spanish]{babel}
\usepackage{amsmath}
\usepackage{amssymb}
\usepackage{amsthm}
\usepackage[utf8]{inputenc}
\usepackage{listings}
\newcommand{\expr}{\langle \mathsf{exp} \rangle}
\newcommand{\var}{\langle \mathsf{var} \rangle}
\newtheorem{definition}{Definición}
\newtheorem{theorem}{Teorema}
\begin{document}
\chapter{Cálculo Lambda}
\label{chap:lambda}

El cálculo lambda es un lenguaje de programacion funcional, el cual subyace y facilita muchos otros lenguajes en general.

\title{Sintaxis}
\maketitle

El cálculo lamba tiene la siguiente sintaxis abtracta \\

\[
\expr ::= \var\,\mid\, \expr \expr \,\mid\, \lambda \var.\expr
\]

donde \( \var \) es un conjunto predefinido de variables infinito. La expresion de la forma <exp> <exp> es llamada aplicacion, dentro de esta expresion tomando como ejemplo e$_{0}$e$_{1}$, e0 es llamado operador e1 el operando. La ultima forma lambda <var>.<exp> es llamada abstracion o expresion lambda.
En Agda, podes definir la sintaxis del cálculo de la siguiente manera: \\

\begin{lstlisting}
  data Expr : Set where
  Var       : V
  App       : Expr -> Expr -> Expr
  Lamb      : V -> Expr -> Expr
\end{lstlisting}
                             
La aplicación es asociativa a la izquierda y en lambda v.e, la subexpresion e se extiende hasta el primer simbolo de paro (como por ejemplo un paréntesis) o hasta el final de la frase. Se dice que la ocurrencia de 'v' tiene como alcance a 'e'.

El conjunto de la variables libres que ocurren una expresión lambda e es definida por: \\


\begin{definition}[Variables Libres]
  \begin{align*}\label{eq:1}
    FV(v) &= {v} \\
    FV(e_{0}e_{1}) &= FV(e_{0}) \cup FV(e_{1}) \\
    FV(\lambda v.e) &= FV(e) - {v} \\
  \end{align*}
\end{definition}

\\

En Agda a las variables libres las podemos expresar de dos formas, una de ellas es como una funcion la cual toma como argumento a una expresión lambda y devuelve una lista de sus variables libres. Para esto defino la estructura de datos de una lista con sus respectivas funciones que voy a necesitar: \\

\begin{lstlisting}
  data List (A : Set) : Set where
  [ ]                 : List A
  _::_                : A -> List A -> List A
\end{lstlisting}
 

\begin{lstlisting}[escapeinside={(*}{*)}]
  _ (*$\in$*) _ : V -> List V -> Bool
  x (*$\in$*) [] = false
  x (*$\in$*) (y::ys) with x==y
  ... | true  = true
  ... | false = x \in ys
\end{lstlisting}
\\
\begin{lstlisting}[escapeinside={(*}{*)}]
  _ +++ _ : List V -> List V -> List V
  ys        +++ [ ] = ys
  (x :: xs) +++ (y :: ys) with x == y
  ... | true = x :: (xs +++ ys)
  ... | false = x :: (y :: ( xs +++ ys))
\end{lstlisting}
\\
\begin{lstlisting}[escapeinside={(*}{*)}]
  _ - _ : List V -> V -> List V
  (x :: xs) - s with x == s
  ... | true = xs - s
  ... | false = x :: (xs - s)
\end{lstlisting}

Y por ultimo defino la funcion que me devuelve las variables libres de una expresión: \\

\begin{lstlisting}[escapeinside={(*}{*)}]
  FreeVList : Expr -> List V
  FreeVList (Var s) = s :: [ ]
  FreeVList (App e(*$_{1} e_{2}$*)) = FreeVList (*e$_{1}$*) +++ FreeVList (*e$_{2}$*)
  FreeVList (Lamb s (*e$_{1}$*)) = FreeVList (*e$_{1}$*) - s
\end{lstlisting}
\\ \\

Otra forma de ver las variables libres es definiendo una relación de dos elementos, donde uno de ellos es una variable y el otro será una expresion lambda. Esta relación sólo estará definida si esa variable es una variable libre de dicha expresión.
Para esto debo escribir todos los casos posibles para la expresiones lambda, con todos sus constructores. La relacion quedaría de la siguiente forma: \\

\begin{lstlisting}[escapeinside={(*}{*)}]
data _ FreeV _ : V -> Expr -> Set where
var : {x y : V} -> x = y ->
         x FreeV (Var y)
appl : {x : V} {e e' : Expr} -> x FreeV e ->
         x FreeV (App e e')
appr : {x : V} {e e' : Expr} -> x FreeV e' ->
         x FreeV (App e e')
abs  : {x y : V} {e : Expr} -> x FreeV e -> (x = y -> (*$\bot$*)) ->
         x FreeV (Lamb y e)
\end{lstlisting}
      

Lo siguiente sería realizar dos pruebas con ambas definiciones. Una de completitud que demuestre que si existe v FreeV e entonces v pertenece a la lista de variables libres de e. Y luego una prueba de correccion, la cual muestra que para toda variable v perteneciente a lista de variables libres de e, existe la relacion v FreeV e.
Las demostraciones también las puedo realizar en Agda:







                             

--------------------------------
                             





Tambien podemos definir e / $\delta$, que es la sustitucion de $\delta$ por cada aparicion de v, de la siguiente manera:

\begin{definition}[Sustitucion]
  \begin{align*}\label{eq:1}
    v / \delta &= \delta v \\
    (e_{0}e_{1})/\delta &= (e_{0}/\delta)(e_{1}/\delta) \\
    (\lambda v.e)/\delta &= \lambda v_{new}.(e/[\delta\dimv:v_{new}]), where  v_{new} \notin Union 
  \end{align*}
\end{definition}

En Agda debo definir la funcion $\Delta$ que toma como argumento una variable y retonar una expresion lambda.
\\
$\Delta$=V -$>$ Expr
\\

Luego la substitucion queda de la siguiente manera:\\

\begin{lstlisting}[escapeinside={(*}{*)}]
 _/ _ : Expr -> (*$\delta$*) -> Expr
Var v / (*$\delta$*)    = (*$\delta$*) v
(App e e')/(*$\delta$*) = App (e/(*$\delta$*)(e'/(*$\delta$*)
(Lamb x e)/(*$\delta$*) = Lamb y (e/((*$\delta$*)+(x,Var y)))
                         where y = fresh x (FreeVSubs (*$\delta$*) (FreeVList e-x))
\end{lstlisting}
 
\begin{proposition}[Proposicion]
  Supongamos que e es una expresión del cálculo lambda, entonces:\\
  (a) Si \delta w = \delta 'w para todo w \in FV(e), entonces e/\delta = e/\delta\\
  (b) e/I_{<var>}=e\\
  (c) FV(e/\delta) = Union FV(\delta w) \\
\end{proposition}



 Renombre o cambio de variable alcanzada:\\
 La operacion de reemplazar la ocurrencia de una expresion lambda $\delta$v.e por $\delta$v$_{new}$.(e/v -$>$ v$_{new}$), donde v$_{new}$ es cualquier variable que no pertenece al conjunto FV(e) - {v}. es llamado renombre o cambio de variable alcanzada.
 Si e' es obtenida a partir de e luego del renombre de cero o mas ocurrencias de subfrases, decimos que que e' es obtenida de e por renombre, o que e es $\alpha$-convertida a e'. Como podremos demostrar luego que es una relacion de equivalencia podemos decir que e y e' son $\alpha$-equivalentes.\\
 En Agda podemos definir la relacion congruencia y la prueba de que es una relación de equivalencia de la siguiente forma:\\ \\
 
\begin{lstlisting}[escapeinside={(*}{*)}]
  data _(*$\neg$*)_ : Expr -> Expr -> Set where
  var : {x : V} ->
    (Var x) (*$\neg$*) (Var x)
  app : {e e' g g' : Expr} -> e (*$\neg$+) e' -> g (*$\neg$*) g' ->
    (App e g) (*$\neg$*) (App e' g')
  lam : {e e' : Expr} {x x' y : V} ->
    (y FreeV (Lamb x e) -> (*$\bot$*)) -> (y FreeV (Lamb x' e') -> (*$\bot$*)) ->
    (e / (idd + (x , Var y))) (*$\neg$*) (e' / (idd + (x' , Var y))) ->
    (Lamb x e) (*$\neg$*) (Lamb x' e')
\end{lstlisting} \\

\begin{lstlisting}[escapeinside={(*}{*)}]
  reflex(*$\neg$*) : (*$\forall$*) (M : Expr) -> M (*$\neg$*) M
  reflex(*$\neg$*) (Var x) = var
  reflex(*$\neg$*) (App e e(*$_{1}$*)) = app (reflex(*$\neg$*) e) (reflex(*$\neg$*) e(*$_{1}$*))
  reflex(*$\neg$*) (Lamb x e) =
\end{lstlisting}
 
\end{document}